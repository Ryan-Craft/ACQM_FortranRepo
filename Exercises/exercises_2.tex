\documentclass{article}
\usepackage{graphicx} % Required for inserting images
\usepackage{amsfonts,amssymb,amsmath,mathbbol,amsthm,amstext,esint,mathtools,nccmath,upgreek}
\graphicspath{ {./Images/} }
\setlength{\jot}{0.5cm}
\DeclareSymbolFontAlphabet{\mathbb}{AMSb}
\DeclareSymbolFontAlphabet{\mathbbl}{bbold}
\usepackage{graphics,graphicx,epsfig,ulem, xcolor, soul}
\usepackage[margin={2cm,2cm}]{geometry}
\usepackage{tikz,pgfplots}
\usepackage{siunitx}
\usepackage{multirow}
\usepackage{array}
\usepackage{physics, amsmath}
\usepackage{braket}
\usepackage{float}
\usepackage{cleveref}
\usepackage{booktabs}
\usepackage[american,siunitx]{circuitikz}
\usepackage{caption}
\usepackage{subcaption}
\usepackage{pgfplotstable}
\pgfplotsset{compat=1.3}
\renewcommand{\vec}[1]{\mathbf{#1}}
\let\oldhat\hat
\renewcommand{\hat}[1]{\oldhat{\mathbf{#1}}}
\AtBeginDocument{
\heavyrulewidth=.08em
\lightrulewidth=.05em
\cmidrulewidth=.03em
\belowrulesep=.65ex
\belowbottomsep=0pt
\aboverulesep=.4ex
\abovetopsep=0pt
\cmidrulesep=\doublerulesep
\cmidrulekern=.5em
\defaultaddspace=.5em
}


\begin{document}
\newpage
\begin{center}
       
       \vspace*{1cm}
        \LARGE
       \textbf{Exercise 2}

       \vspace{0.5cm}
        PHYS4000
        Advanced Computational Quantum Mechanics
            
       \vspace{1.5cm}

       \text{Ryan Craft}\bigskip

       Unit Coordinated by\\
       Dr. Igor Bray
            
       \vspace{0.8cm}
     
            
       
        \today
            
\end{center}
\newpage


\section{Question 1}

    Calculate the expectation value $\braket{\hat{S}_z}$ (will be using bold \& hat notation for operators because
    it looks nice in latex) for $\ket{\psi}\;=\;\ket{z_-}$:
    \medskip

    Starting with the definition for the expectation value:

    \begin{gather}
        \braket{\hat{S_z}} = \braket{\psi|\hat{S_z}|\psi}\\
        \braket{\psi|\hat{S_z}|\psi} = \frac{1}{2}(|\braket{\psi|z_+}|^2 - |\braket{\psi|z_-}|^2)
    \end{gather}

    for a general state $\ket{\psi}$, which we substitute with $\ket{z_-}$ to get

    \begin{gather}
        \braket{z_-|\hat{S_z}|z_-} = \frac{1}{2}(|\braket{z_-|z_+}|^2 - |\braket{z_-|z_-}|^2)
    \end{gather}

    We are aware that $\braket{z_{\pm}|z_{\mp}} = 0$ and  $\braket{z_{\pm}|z_{\pm}} = 1$, subsequently

    \begin{gather*}
        \braket{z_-|\hat{S_z}|z_-} = \frac{1}{2}(-1)^2
    \end{gather*}
    \begin{gather}
        \braket{z_-|\hat{S_z}|z_-} = \frac{1}{2}
    \end{gather}

    We expect this because its essentially asking what the expectation that we get a $\ket{z_-}$ state
    from an SG\_z experiment is. And we know from Exp 1 that it ought to be $1/2$.


\section{Question 2}

    As in question one, for a general state $\ket{\psi}$, the expectation value $\hat{S_z}$

    \begin{gather}
        \braket{\hat{S_z}} = \braket{\psi|\hat{S_z}|\psi}\\
        \braket{\hat{S_z}} = \frac{1}{2}(|\braket{\psi|z_+}|^2 - |\braket{\psi|z_-}|^2)
    \end{gather}

    There are two key terms $\braket{\psi|z_+}$ and $\braket{\psi|z_-}$ which we compute:

    \begin{gather}
        \braket{\psi|z_+} = -\frac{i}{\sqrt{3}}\braket{z_+|z_+} + 0 = -\frac{i}{\sqrt{3}}\\
        \braket{\psi|z_-} = 0 + \sqrt{\frac{2}{3}}\braket{z_-|z_-} = \sqrt{\frac{2}{3}}
    \end{gather}

    Which we apply to $\braket{\hat{S_z}}$

    \begin{gather}
        \braket{\hat{S_z}} = \frac{1}{2}(|-\frac{i}{\sqrt{3}}|^2 - |\sqrt{\frac{2}{3}}|^2)\\
        \braket{\hat{S_z}} = \frac{1}{2}\left( \frac{1}{3} - \frac{2}{3} \right)\\
        \braket{\hat{S_z}} = \frac{1}{6}
    \end{gather}



\section{Question 3}

    We start with the knowledge that a projection operator satisfies $P^2 = P$,
    and we apply this to $I_y$:

    \begin{gather}
        I_y = \ket{y_+}\bra{y_+} + \ket{y_-}\bra{y_-}
    \end{gather}
    \begin{gather*}
        I_y^2 = (\ket{y_+}\bra{y_+} + \ket{y_-}\bra{y_-})^2\\
        I_y^2 = \ket{y_+}\braket{y_+|y_-}\bra{y_-} + \ket{y_-}\bra{y_-|y_+}\bra{y_+}
        + (\ket{y_+}\bra{y_+})^2 + (\ket{y_-}\bra{y_-})^2\\
        I_y^2 = 0 + 0 + \ket{y_+}\braket{y_+|y_+}\bra{y_+} + \ket{y_-}\braket{y_-|y_-}\bra{y_-}\\
        I_y^2 = \ket{y_+}\bra{y_+} + \ket{y_-}\bra{y_-}
    \end{gather*}\
    \begin{gather*}
        I_y^2 = \ket{y_+}\bra{y_+} + \ket{y_-}\bra{y_-} = I_y
    \end{gather*}

    The above shows that $I_y^2 = I_y$ and thus satisfies the condition of a projection operator. 


\section{Question 4}

    Showing that $I_y=I_z$ using 5.12.

    5.12:

    \begin{gather}
        \ket{y_\pm} = \frac{1}{\sqrt{2}}\ket{z_+} \pm \frac{i}{\sqrt{2}}\ket{z_-}
    \end{gather}

    Substitute directly into $I_y$:

    \begin{gather*}
        I_y = \ket{y_+}\bra{y_+} + \ket{y_-}\bra{y_-}\\
        I_y = \frac{1}{\sqrt{2}}\ket{z_+}\left(\frac{1}{\sqrt{2}}\bra{z_+}-\frac{i}{\sqrt{2}}\bra{z_-}\right)\\
        +     \frac{i}{\sqrt{2}}\ket{z_-}\left(\frac{1}{\sqrt{2}}\bra{z_+}-\frac{i}{\sqrt{2}}\bra{z_-}\right)\\
        +     \frac{1}{\sqrt{2}}\ket{z_+}\left(\frac{1}{\sqrt{2}}\bra{z_+}+\frac{i}{\sqrt{2}}\bra{z_-}\right)\\
        +     \frac{i}{\sqrt{2}}\ket{z_-}\left(\frac{1}{\sqrt{2}}\bra{z_+}+\frac{i}{\sqrt{2}}\bra{z_-}\right)\\
    \end{gather*}
    \begin{gather*}
       I_y = \ket{z_+}\bra{z_-} - \frac{i}{2}\ket{z_+}\bra{z_-}\\
        + \frac{i}{2}\ket{z_-}\bra{z_+} + \frac{1}{2}\ket{z_-}\bra{z_-}\\
        + \frac{1}{2}\ket{z_+}\bra{z_+} + \frac{i}{2}\ket{z_+}\bra{z_-}\\
        - \frac{i}{2}\ket{z_-}\bra{z_+} - \frac{1}{2}\ket{z_-}\bra{z_-}\\
    \end{gather*}
    \begin{gather}
        I_y = \ket{z_+}\bra{z_+} + \ket{z_-}\bra{z_-} = I_z
    \end{gather}


\section{Question 5}

    5.15:

    \begin{gather*}
        0 = |\braket{z_-|x_+}|^2|\braket{x_+|z_+}|^2 + |\braket{z_-|x_-}|^2|\braket{x_-|z_+}|^2\\
          + \braket{z_-|x_+}\braket{x_+|z_+}\braket{z_-|x_-}^*\braket{x_-|z_+}^*\\
          + \braket{z_-|x_+}^*\braket{x_+|z_+}^*\braket{z_-|x_-}\braket{x_-|z_+}
    \end{gather*}

    We can note that there are really only four terms, and their respective conjugates.
    If we substitute $\ket{x_\pm} = \frac{1}{\sqrt{2}}(\ket{z_+} \pm \ket{z_+})$ into the four repeating terms and evaluate them,
    we can simplify the RHS in terms of z.

    \begin{gather}
        \braket{z_-|x_+} = \bra{z_-}\left( \frac{1}{\sqrt{2}}\ket{z_+} + \frac{1}{\sqrt{2}}\ket{z_-} \right)\\
        \braket{x_+|z_+} = \left( \frac{1}{\sqrt{2}}\ket{z_+} + \frac{1}{\sqrt{2}}\ket{z_-} \right)\ket{z_+}\\
        \braket{z_-|x_-} = \bra{z_-}\left( \frac{1}{\sqrt{2}}\ket{z_+} - \frac{1}{\sqrt{2}}\ket{z_-} \right)\\
        \braket{x_-|z_+} = \left( \frac{1}{\sqrt{2}}\ket{z_+} - \frac{1}{\sqrt{2}}\ket{z_-} \right)\ket{z_+}
    \end{gather}

    Which evaluate respectively to

    \begin{gather}
        \braket{z_-|x_+} = \frac{1}{\sqrt{2}}\:,
        \braket{x_+|z_+} = \frac{1}{\sqrt{2}}\\
        \braket{z_-|x_-} = -\frac{1}{\sqrt{2}}\:,
        \braket{x_-|z_+} = \frac{1}{\sqrt{2}}
    \end{gather}

    Substituting the evaluated amplitudes into 5.15 we get

    \begin{gather*}
        0 = |\frac{1}{\sqrt{2}}|^2|\frac{1}{\sqrt{2}}|^2 + |-\frac{1}{\sqrt{2}}|^2|\frac{1}{\sqrt{2}}|^2\\
          + (\frac{1}{\sqrt{2}})(\frac{1}{\sqrt{2}})(-\frac{1}{\sqrt{2}})^*(\frac{1}{\sqrt{2}})^*\\
          + (\frac{1}{\sqrt{2}})^*(\frac{1}{\sqrt{2}})^*(-\frac{1}{\sqrt{2}})(\frac{1}{\sqrt{2}})\\
          = \frac{1}{4} + \frac{1}{4} - \frac{1}{4} - \frac{1}{4} = 0
    \end{gather*}

    RHS = 0 and LHS = RHS.

    


\section{Question 6}

Show that $(\hat{A}^{\dagger})^{\dagger} = \hat{A}$.

For arbitrary $\ket{\chi}$

\begin{gather}
    \braket{\psi|\hat{A}^\dagger|\chi} = \braket{\chi|\hat{A}|\psi}^*
\end{gather}

It follows that

\begin{gather}
    \braket{\psi|\hat{A}^\dagger|\chi}^* = \braket{\chi|\hat{A}|\psi}
\end{gather}

These two facts imply that

\begin{gather}
    \braket{\psi|\hat{A}^\dagger|\chi}^* = \braket{\chi|(\hat{A}^\dagger)^\dagger|\psi}\\
    \braket{\psi|\hat{A}^\dagger|\chi}^* = \braket{\chi|\hat{A}|\psi} = \braket{\chi|(\hat{A}^\dagger)^\dagger|\psi}\\
    \therefore (\hat{A}^{\dagger})^{\dagger} = \hat{A}
\end{gather}

\vspace{2cm}

From this we can state:


\begin{gather}
    \bra{\phi}\hat{A} = \bra{\phi}(\hat{A}^{\dagger})^{\dagger}\\
    \bra{\phi}(\hat{A}^{\dagger})^{\dagger} = \bra{\hat{A}^{\dagger}\phi}/
\end{gather}
\begin{center}
    Thus, $\bra{\phi}\hat{A} = \bra{\hat{A}^{\dagger}\phi}$
\end{center}




\section{Question 7}

    Show that $(\hat{A}\hat{B})^\dagger = \hat{B}^\dagger \hat{A}^\dagger$

    \begin{gather}
        \braket{\phi | \hat{A}\hat{B}\psi}\\
        = \braket{\hat{A}^\dagger \phi | \hat{B}\psi}\\
        = \braket{\hat{B}^\dagger\hat{A}^\dagger \phi | \hat{B}\psi}
    \end{gather}


\section{Question 8}

If $\hat{A}$ is Hermitian, then by definition $\hat{A}^\dagger = \hat{A}$.

The eigenvalue equation:

\begin{gather}
    \hat{A}\ket{\psi} = a\ket{\psi}
\end{gather}

Lets pre-multiply by a generic $\bra{\psi}$.

\begin{gather}
    \braket{\psi|\hat{A}|\psi} = a \braket{\psi|\psi}
\end{gather}

If we do the same with the adjoint operator;

\begin{gather}
    \braket{\psi|\hat{A}^\dagger|\psi} = a^* \braket{\psi|\psi}
\end{gather}

If we know that $\hat{A}^\dagger = \hat{A}$, then that implies that:

\begin{gather}
    a \braket{\psi|\psi} = a^* \braket{\psi|\psi}\\
    a = a^*
\end{gather}

The only condition for which $a = a^*$ is if $a \in \mathbb{R}$. Thus the eigenvalues of Hermitian operators are real.

\end{document}