\documentclass{article}
\usepackage{graphicx} % Required for inserting images
\usepackage{amsfonts,amssymb,amsmath,mathbbol,amsthm,amstext,esint,mathtools,nccmath,upgreek}
\graphicspath{ {./Images/} }
\setlength{\jot}{0.5cm}
\DeclareSymbolFontAlphabet{\mathbb}{AMSb}
\DeclareSymbolFontAlphabet{\mathbbl}{bbold}
\usepackage{graphics,graphicx,epsfig,ulem, xcolor, soul}
\usepackage[margin={2cm,2cm}]{geometry}
\usepackage{tikz,pgfplots}
\usepackage{siunitx}
\usepackage{multirow}
\usepackage{array}
\usepackage{physics, amsmath}
\usepackage{float}
\usepackage{cleveref}
\usepackage{booktabs}
\usepackage[american,siunitx]{circuitikz}
\usepackage{caption}
\usepackage{subcaption}
\usepackage{pgfplotstable}
\pgfplotsset{compat=1.3}
\renewcommand{\vec}[1]{\mathbf{#1}}
\let\oldhat\hat
\renewcommand{\hat}[1]{\oldhat{\mathbf{#1}}}
\AtBeginDocument{
\heavyrulewidth=.08em
\lightrulewidth=.05em
\cmidrulewidth=.03em
\belowrulesep=.65ex
\belowbottomsep=0pt
\aboverulesep=.4ex
\abovetopsep=0pt
\cmidrulesep=\doublerulesep
\cmidrulekern=.5em
\defaultaddspace=.5em
}


\begin{document}
\newpage
\begin{center}
       
       \vspace*{1cm}
        \LARGE
       \textbf{Assignment 2a}

       \vspace{0.5cm}
        PHYS4000
        Advanced Computational Quantum Mechanics
            
       \vspace{1.5cm}

       \text{Ryan Craft}\bigskip

       Unit Coordinated by\\
       Dr. Igor Bray
            
       \vspace{0.8cm}
     
            
       
       May 2023
            
\end{center}
\newpage



\section{Question 1}

Equation 2.17 is written as:

    \begin{gather}
        \frac{d^2u_l(r)}{dr^2} + 2(E+\frac{Z}{r}) u_l(r) = l(l+1)\frac{u_l(r)}{r^2}
    \end{gather}

    First we need to make $\rho$ the subject of the function which is done through the substitution of
    $\rho=kr$.

    \begin{gather}
        k^2 \frac{d^2u_l(\rho)}{d\rho^2} + 2(-\frac{k^2}{8}+\frac{Z}{\rho k})u_l(\rho) =  l(l+1)\frac{u_l(\rho)k^2}{\rho^2}
    \end{gather}

    Where we have substituted for $E$ by rearranging the definition of $k=2\sqrt{-2E}$. 
    We further simplify the differential Equation to the following by cancelling rho and multiplying the bracket:

    \begin{gather}
        \frac{d^2u_l(\rho)}{d\rho^2} - (\frac{1}{4}-\frac{2Z}{\rho k})u_l(\rho) =  l(l+1)\frac{u_l(\rho)}{\rho^2}
    \end{gather}

    In finding the solutions to the differential equation we rearrange it into the form\dots

    \begin{gather}
        \frac{d^2u_l(\rho)}{d\rho^2} - (\frac{1}{4}-\frac{2Z}{\rho k})u_l(\rho) - l(l+1)\frac{u_l(\rho)}{\rho^2} = 0
    \end{gather}


    The question presents the substitution $u_l(\rho)=\rho^{l+1}\exp{-\rho/2}\omega(\rho)$, which forms\dots

    \begin{gather}
        \frac{d^2}{d\rho^2}[\rho^{l+1}\exp{-\rho/2}\omega(\rho)] - (\frac{1}{4}-\frac{2Z}{\rho k})(\rho^{l+1}\exp{-\rho/2}\omega(\rho))
        - l(l+1)\rho^{l-1}\exp{-\rho/2}\omega(\rho) = 0
    \end{gather}
    \medskip
    
    {\centering
    \begin{gather*}
        (l+1)[(l\rho^{l-1}\exp{-\rho/2}-\frac{1}{2}\rho^{l+1}\exp{-\rho/2})\omega(\rho) + \rho^l \exp{-\rho/2}\frac{d\omega(\rho)}{d\rho}]\\
        - \frac{1}{2}\{  [(l+1) \rho^l \exp{-\rho/2} 
        - \frac{1}{2}\rho^{l+1}\exp{-\rho/2}]\omega(\rho) 
        + \frac{d\omega(\rho)}{d\rho^2}\rho^{l+1}\exp{-\rho/2}\}\\ 
        +[(l+1)\rho^l\exp{-\rho/2}-\frac{1}{2}\rho^{l+1}\exp{-\rho/2}]\frac{d\omega(\rho)}{d\rho} + \frac{d^2\omega(\rho)}{d\rho^2}\rho^{l+1}\exp{-\rho/2}\\
        - (\frac{1}{4}-\frac{2Z}{k\rho})\rho^{l+1}\exp{-\rho/2}\omega(\rho) - l(l+1)\rho^{l-1}\exp{-\rho/2}\omega(\rho)
        = 0
    \end{gather*}
    }

    and simplifies to

    \begin{gather}
        \frac{d^2\omega(\rho)}{d\rho^2}\rho^{l+1} + \frac{d\omega(\rho)}{d\rho}[2(l+1)\rho^l - \rho^{l+1}] - \omega(\rho)[(l+1)\rho^l - \frac{2Z}{k}\rho^l]
        =0
    \end{gather}

    We now eliminate $\rho^l$ on all terms to get:

    \begin{gather}
        \frac{d^2\omega(\rho)}{d\rho^2}\rho + [2(l+1) - \rho]\frac{d\omega(\rho)}{d\rho} - [(l+1) - \frac{2Z}{k}]\omega(\rho)
        =0
    \end{gather}
    
    This differential equation has the form:

    \begin{gather}
        z\frac{d^2w}{dz^2} + (b-z)\frac{dw}{dz} - aw = 0
    \end{gather}

    Which has solutions in the confluent hypergeometric function;

    \begin{gather}
        F(a,b;\rho) = 1 + \frac{a}{b}\frac{\rho}{1!} + \frac{a}{b}\frac{(a+1)}{(b+1)}\frac{\rho^2}{2!} + \dots
    \end{gather}
    
    In this case, the function will truncate so long as $a=1-n$ for $n=1,2,3$,
    in this way, after the $n+1$ term of the series and all subsequent terms must simplify to zero.

    In this case then by comparing the equation we have with the general form, we can see that

    \begin{align}
        a &= l+1 - \frac{2Z}{k}\\
        b &= 2(l+1)
    \end{align}

    And knowing that the differential equation is has the general solution and that the series terminates, we know that this set of
    equations satisfies the conditions for a solution of $\omega(\rho)$.



\section{Question 2}

    The equation:

    \begin{gather}
        F(a,b;\rho) = 1 + \frac{a}{b}\frac{\rho}{1!} + \frac{a}{b}\frac{(a+1)}{(b+1)}\frac{\rho^2}{2!} + \dots
    \end{gather}

    Terminates for $a = 1 - n$, for $n \in \{1,2,3,\dots\}$. 
    This prevents the series from growing exponentially in $\rho$.\\

    In our case, we have $a = 1 + l - \frac{2Z}{k}$, which can be rewritten as $a = 1 - (\frac{2Z}{k}-l)$.\\
    This makes our $n = \frac{2Z}{k}-l$ and subsequently $n+l=\frac{2Z}{k}$.\\ 
    \smallskip
    We then write $n=\frac{2Z}{k}, \;\forall\; n \in \{1 + l, 2 + l, 3 + l, \dots$\}.\\

\section{Question 3}

    2.28 states:

    \begin{gather}
        \int_{E}^{yee}
    \end{gather}
    















\end{document}