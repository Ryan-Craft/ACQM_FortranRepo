\documentclass{article}
\usepackage{graphicx, multicol} % Required for inserting images
\usepackage{amsfonts,amssymb,amsmath,mathbbol,amsthm,amstext,esint,mathtools,nccmath,upgreek}
\graphicspath{ {../Images/} }
\DeclareSymbolFontAlphabet{\mathbb}{AMSb}
\DeclareSymbolFontAlphabet{\mathbbl}{bbold}
\usepackage{graphics,graphicx,epsfig, xcolor, soul}
\usepackage[normalem]{ulem}
\usepackage[margin={2cm,2cm}]{geometry}
\usepackage{tikz,pgfplots, hyperref}
\usepackage{siunitx}
\usepackage{multirow}
\usepackage{array}
\usepackage{physics}
\usepackage{float}
\usepackage{cleveref}
\usepackage{booktabs}
\usepackage[american,siunitx]{circuitikz}
\usepackage{caption}
\usepackage{subcaption}
\newcolumntype{M}[1]{>{\centering\arraybackslash}m{#1}}
\usepackage{pgfplotstable}
\pgfplotsset{compat=1.3}
\renewcommand{\vec}[1]{\mathbf{#1}}
\let\oldhat\hat
\renewcommand{\hat}[1]{\oldhat{\mathbf{#1}}}

\AtBeginDocument{
\RenewCommandCopy\qty\SI
\heavyrulewidth=.08em
\lightrulewidth=.05em
\cmidrulewidth=.03em
\belowrulesep=.65ex
\belowbottomsep=0pt
\aboverulesep=.4ex
\abovetopsep=0pt
\cmidrulesep=\doublerulesep
\cmidrulekern=.5em
\defaultaddspace=.5em
\parindent=0pt
\parskip=6pt plus 2pt
}
\usepackage[authordate,natbib,maxcitenames=3]{biblatex-chicago}
\addbibresource{bib2.bib}
\setcounter{tocdepth}{5}
\setcounter{secnumdepth}{5}
\newcommand\simpleparagraph[1]{%
	\stepcounter{paragraph}\paragraph*{\theparagraph\quad{}#1}}



\begin{document}
\noindent

	\title{Computational Assignment 1: Using the Laguerre Basis to get Hydrogen Energies and Radial Wavefuctions}
    \maketitle 
    
    \section{Problem 1}
    
    Solutions to the Schrodinger equation for the Hydrogen atom come in the separable form:
    \begin{gather}
    	\Phi_{nlm}(\vec{r}) =  \Phi_{nl}(r)*Y_{l}^m(\hat{\vec{r}})
    \end{gather}
    
    Where $\Phi_{nl}(r)$ are the spherically symmetric radially dependent parts of the wavefunction and $Y_{l}^m(\hat{\vec{r}})$ are the Spherical Harmonics, for the quantum numbers n,l and m, representing the principal, angular and magnetic quantum numbers. 
    
    These wavefunctions can be represented as:
    \begin{gather}
    	\Phi_{nlm}(\vec{r}) =  \Phi_{nl}(r)*Y_{l}^m(\hat{\vec{r}})
    \end{gather}
    
    Analytical solutions to the bound state radial part of the hydrogen atom are completely known, the first few relevant ones for the rest of this report follow:
    \large
    \begin{center}
    	\begin{tabular}{lcc}\toprule
    		n & l & $\Phi_{nl}(r)$ \\ \bottomrule
    		1 & 0  &   $2re^{-r}$          \\
    		&&\\
    		2 & 0  & $\frac{r}{\sqrt{2}}(1-\frac{r}{2})e^{-r/2} $  \\
    		&&\\
    		2 & 1  & $ \frac{r^2}{\sqrt{24}}(1-\frac{2r}{3} + \frac{2r^2}{27})e^{-r/3}  $  \\
    		&&\\
    		3 & 0  & $\frac{2r}{\sqrt{27}}(1-\frac{2r}{3}+\frac{2r^2}{27})e^{-r/3}$   \\
    		&&\\
    		3 & 1  & $\frac{8r^2}{27\sqrt{6}}(1-\frac{r}{6})e^{-r/3}$ \\ 
    		&&\\
    		4 & 1  & $\frac{r^2}{64\sqrt{15}}(\frac{r^2}{4}-5r+20)e^{-r/4}$ \\ \bottomrule
    	\end{tabular}
    \end{center}
    \normalsize
    
    If we choose a set of basis functions $\phi_{j}$ for $k={1,2,\; ...\;, \infty}$ which form a complete basis on the Hilbert space, defined as:
    \large
    \begin{gather}
    	\braket{\vec{r}}{\phi_j} = \frac{1}{r}\phi_{k_j,l_j}(r)Y^{m_j}_{l_j}(\hat{\vec{r}})
    \end{gather}
    \normalsize
    
    These basis function can be used to recover high order approximations to the true radial wavefunction through a sum over a finite number of the basis functions in the following way:
   
    \large
    \begin{gather}
    	\ket{\Phi_i} = \sum_{j}^{N} c_ji \ket{\phi_j}
    \end{gather}
    \normalsize
    
    We need to generate the basis functions for an arbitrary sized basis $N$.
    
    
    
    \subsection{Problem 2}
    
    \textcolor{red}{INTRODUCE THESE IDEAS}
    
    Where the Real wavefunction can be recovered as a linear combination of all of these vectors:
    \large
    \begin{gather}
    	\ket{\Phi} = \sum_{j}c_j\ket{\phi_j}
    \end{gather}
    \normalsize
    
    If we make this substitution into the Schrodinger equation:
    \large
    \begin{gather}
    	\sum_{j}c_j\vec{H}\ket{\phi_j} = E\sum_{j}c_j\ket{\phi_j}\\
    	\sum_{j}c_j\bra{\phi_i}\vec{H}\ket{\phi_j} = E\sum_{j}c_j\bra{\phi_i}\ket{\phi_j}
    \end{gather}
    \normalsize
    
    As a matrix equation then becomes:
    \large
    \begin{gather}
    	\sum_{j}c_{ji}\vec{H_{ij}} = E_i\sum_{j}\vec{B_{ij}}c_{ji}
    \end{gather}
    \normalsize


    In the compute program this is performed over a finite basis N, thus we get:
    
	\large
	\begin{gather}
		\sum_{j}^{N}c_{ji}\vec{H_{ij}} = E_i\sum_{j}^{N}\vec{B_{ij}}c_{ji}
	\end{gather}
	\normalsize
 		
	
\end{document}












